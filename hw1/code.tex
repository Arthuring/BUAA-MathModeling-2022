\chapter{代码附录}
\section{sympy解法}
\subsection{代码}
\begin{python}
# -- coding: utf-8 --
import sympy as sp

l = sp.var('l')
exp = (l - sp.Rational(6, 10)) * (l - sp.Rational(9, 10)) * (l - sp.Rational(7, 10)) - sp.Rational(9, 100) * (
        l - sp.Rational(9, 10)) - sp.Rational(1, 100) * (l - sp.Rational(7, 10))
expp = sp.simplify(exp)

print("特征方程为:{}".format(expp))
k = sp.var('k', positive=True, integer=True)

L = sp.Matrix([[sp.Rational(6, 10), sp.Rational(1, 10), sp.Rational(3, 10)],
               [sp.Rational(1, 10), sp.Rational(9, 10), 0],
               [sp.Rational(3, 10), 0, sp.Rational(7, 10)]])
x_0 = sp.Matrix([[2], [1], [1]])
val = L.eigenvals()
print("特征值:{}".format(val))
vec = L.eigenvects()
print("特征向量:{}".format(vec))

P, D = L.diagonalize()  # 相似对角化
print("相似对角化后P:")
print(P)
print("相似对角化后D:")
print(D)

Lk = P @ (D ** k) @ (P.inv())  # 计算L的k次方
F = Lk @ sp.Matrix([[2], [1], [1]])  # 求出最终解x_k
print("最终解为:\n{}".format(F))
# ---------计算稳定解-------------------
Pinv = P.inv()  # 计算P的逆矩阵
c = (Pinv @ x_0)[0]  # 计算c的值
print("c={}".format(c))
print("稳定解为:\n{}".format(c * sp.Matrix([[1],[1],[1]])))
\end{python}
\subsection{输出}
\begin{python}
D:\program\Anaconda\python.exe D:/courses/gradeThree/mathModeling/hw1/solution_sympy.py
特征方程为:l**3 - 11*l**2/5 + 149*l/100 - 29/100
特征值:{1: 1, 3/5 - sqrt(7)/10: 1, sqrt(7)/10 + 3/5: 1}
特征向量:[(1, 1, [Matrix([
[1],
[1],
[1]])]), (3/5 - sqrt(7)/10, 1, [Matrix([
[-(7 + 3*sqrt(7))/(2*sqrt(7) + 7)],
[  3/(100*(3/50 + 3*sqrt(7)/100))],
[                               1]])]), (sqrt(7)/10 + 3/5, 1, [Matrix([
[-(-7 + 3*sqrt(7))/(-7 + 2*sqrt(7))],
[    3/(100*(3/50 - 3*sqrt(7)/100))],
[                                 1]])])]
相似对角化后P:
Matrix([[1, -sqrt(7)/3 - 1/3, -1/3 + sqrt(7)/3], [1, -2/3 + sqrt(7)/3, -sqrt(7)/3 - 2/3], [1, 1, 1]])
相似对角化后D:
Matrix([[1, 0, 0], [0, 3/5 - sqrt(7)/10, 0], [0, 0, sqrt(7)/10 + 3/5]])
最终解为:
Matrix([[3*(3/5 - sqrt(7)/10)**k*(-sqrt(7)/3 - 1/3)/(28/3 - 2*sqrt(7)/3) - (-6 + 3*sqrt(7))*(3/5 - sqrt(7)/10)**k*(-sqrt(7)/3 - 1/3)/(-28 + 2*sqrt(7)) + 2*(3/5 - sqrt(7)/10)**k*(3 + 3*sqrt(7))*(-sqrt(7)/3 - 1/3)/(-28 + 2*sqrt(7)) + (-1/3 + sqrt(7)/3)*(-2*sqrt(7)/21 - 1/6)*(sqrt(7)/10 + 3/5)**k + (-1/3 + sqrt(7)/3)*(1/3 - sqrt(7)/42)*(sqrt(7)/10 + 3/5)**k + 2*(-1/3 + sqrt(7)/3)*(-1/6 + 5*sqrt(7)/42)*(sqrt(7)/10 + 3/5)**k + 4/3], [2*(-2/3 + sqrt(7)/3)*(3/5 - sqrt(7)/10)**k*(3 + 3*sqrt(7))/(-28 + 2*sqrt(7)) - (-6 + 3*sqrt(7))*(-2/3 + sqrt(7)/3)*(3/5 - sqrt(7)/10)**k/(-28 + 2*sqrt(7)) + 3*(-2/3 + sqrt(7)/3)*(3/5 - sqrt(7)/10)**k/(28/3 - 2*sqrt(7)/3) + 2*(-1/6 + 5*sqrt(7)/42)*(-sqrt(7)/3 - 2/3)*(sqrt(7)/10 + 3/5)**k + (1/3 - sqrt(7)/42)*(-sqrt(7)/3 - 2/3)*(sqrt(7)/10 + 3/5)**k + (-sqrt(7)/3 - 2/3)*(-2*sqrt(7)/21 - 1/6)*(sqrt(7)/10 + 3/5)**k + 4/3], [2*(3/5 - sqrt(7)/10)**k*(3 + 3*sqrt(7))/(-28 + 2*sqrt(7)) - (-6 + 3*sqrt(7))*(3/5 - sqrt(7)/10)**k/(-28 + 2*sqrt(7)) + 3*(3/5 - sqrt(7)/10)**k/(28/3 - 2*sqrt(7)/3) + (-2*sqrt(7)/21 - 1/6)*(sqrt(7)/10 + 3/5)**k + (1/3 - sqrt(7)/42)*(sqrt(7)/10 + 3/5)**k + 2*(-1/6 + 5*sqrt(7)/42)*(sqrt(7)/10 + 3/5)**k + 4/3]])
c=4/3
稳定解为:
Matrix([[4/3], [4/3], [4/3]])

Process finished with exit code 0

\end{python}
\section{numpy解法求数值解}
\subsection{代码}
\begin{python}
    # -- coding: utf-8 --
# -- coding: utf-8 --
import sympy as sp
import numpy as np

l = sp.var('l')
exp = (l - sp.Rational(6, 10)) * (l - sp.Rational(9, 10)) * (l - sp.Rational(7, 10)) - sp.Rational(9, 100) * (
        l - sp.Rational(9, 10)) - sp.Rational(1, 100) * (l - sp.Rational(7, 10))
expp = sp.simplify(exp)

print("特征方程为:{}".format(expp))
k = sp.var('k', positive=True, integer=True)

L = sp.Matrix([[0.6, 0.1, 0.3],
           [0.1, 0.9, 0],
           [0.3, 0, 0.7]])

Lp = np.mat([[0.6, 0.1, 0.3],
           [0.1, 0.9, 0],
           [0.3, 0, 0.7]])
x_0 = sp.Matrix([[2], [1], [1]])
val, vec = np.linalg.eig(Lp)
print("特征值:{}".format(val))

print("特征向量:{}".format(vec))

P, D = L.diagonalize()  # 相似对角化
print("相似对角化后P:")
print(P)
print("相似对角化后D:")
print(D)

Lk = P @ (D ** k) @ (P.inv())  # 计算L的k次方
F = Lk @ np.array([[2], [1], [1]])  # 求出最终解x_k
print("最终解为:\n{}".format(F))
\end{python}
\subsection{输出}
\begin{python}
    特征方程为:l**3 - 11*l**2/5 + 149*l/100 - 29/100
特征值:[0.33542487 1.         0.86457513]
特征向量:[[ 0.76505532 -0.57735027  0.28523152]
 [-0.13550992 -0.57735027 -0.8051731 ]
 [-0.6295454  -0.57735027  0.51994159]]
相似对角化后P:
Matrix([[-0.765055323929465, -0.285231516480645, 0.577350269189626], [0.135509922732534, 0.805173104063772, 0.577350269189626], [0.629545401196931, -0.519941587583127, 0.577350269189626]])
相似对角化后D:
Matrix([[0.335424868893541, 0, 0], [0, 0.864575131106459, 0], [0, 0, 1.00000000000000]])
最终解为:
Matrix([[0.585309648672818*0.335424868893541**k + 0.0813570179938488*0.864575131106459**k + 1.33333333333333*1.0**k], [-0.103672587831795*0.335424868893541**k - 0.229660745501538*0.864575131106459**k + 1.33333333333333*1.0**k], [-0.481637060841023*0.335424868893541**k + 0.14830372750769*0.864575131106459**k + 1.33333333333333*1.0**k]])

\end{python}

\section{numpy解法求解的数值}
\subsection{代码}
\begin{python}
# -- coding: utf-8 --
import numpy as np
import pandas as pd
import pylab as plt

L = np.mat([[0.6, 0.1, 0.3],
           [0.1, 0.9, 0],
           [0.3, 0, 0.7]])

x_0 = np.mat([[2], [1], [1]])

val,vec = np.linalg.eig(L)
print("特征值:{}".format(val))
print("特征向量:{}".format(vec))

x_1 = []
x_2 = []
x_3 = []
x_k = []
limt = []
temp = x_0

for i in range(20):
    temp = L.dot(temp)
    x_k.append(temp.tolist())

    print("k = {}".format(i+1))
    print("x_k = {}".format(temp))
index = [i for i in range(20)]
df = pd.DataFrame(data={'x_k': x_k})
df.to_csv('./data.csv', sep=';')
temp = x_0
for i in range (100):
    temp = L.dot(temp)
    x_1.append(temp.tolist()[0][0])
    x_2.append(temp.tolist()[1][0])
    x_3.append(temp.tolist()[2][0])
    limt.append(4.0/3.0)

plt.xlabel("k")
plt.ylabel("y")
plt.title(r'Numerical solution of each component of $x_k$')
plt.plot(x_1,label=r'$x_k[1]$')
plt.plot(x_2,label=r'$x_k[2]$')
plt.plot(x_3,label=r'$x_k[3]$')
plt.plot(limt,'k--')
plt.legend(loc=1)
plt.text(60,1.35,r'$y = 4/3$')
plt.show()
\end{python}
\subsection{输出}
\begin{python}
特征值:[0.33542487 1.         0.86457513]
特征向量:[[ 0.76505532 -0.57735027  0.28523152]
 [-0.13550992 -0.57735027 -0.8051731 ]
 [-0.6295454  -0.57735027  0.51994159]]
k = 1
x_k = [[1.6]
 [1.1]
 [1.3]]
k = 2
x_k = [[1.46]
 [1.15]
 [1.39]]
k = 3
x_k = [[1.408]
 [1.181]
 [1.411]]
k = 4
x_k = [[1.3862]
 [1.2037]
 [1.4101]]
k = 5
x_k = [[1.37512]
 [1.22195]
 [1.40293]]
k = 6
x_k = [[1.368146]
 [1.237267]
 [1.394587]]
k = 7
x_k = [[1.3629904]
 [1.2503549]
 [1.3866547]]
k = 8
x_k = [[1.35882614]
 [1.26161845]
 [1.37955541]]
k = 9
x_k = [[1.35532415]
 [1.27133922]
 [1.37333663]]
k = 10
x_k = [[1.3523294 ]
 [1.27973771]
 [1.36793289]]
k = 11
x_k = [[1.34975128]
 [1.28699688]
 [1.36325184]]
k = 12
x_k = [[1.34752601]
 [1.29327232]
 [1.35920167]]
k = 13
x_k = [[1.34560334]
 [1.29869769]
 [1.35569897]]
k = 14
x_k = [[1.34394146]
 [1.30338825]
 [1.35267028]]
k = 15
x_k = [[1.34250479]
 [1.30744358]
 [1.35005164]]
k = 16
x_k = [[1.34126272]
 [1.3109497 ]
 [1.34778758]]
k = 17
x_k = [[1.34018888]
 [1.313981  ]
 [1.34583012]]
k = 18
x_k = [[1.33926046]
 [1.31660179]
 [1.34413775]]
k = 19
x_k = [[1.33845778]
 [1.31886765]
 [1.34267456]]
k = 20
x_k = [[1.3377638 ]
 [1.32082667]
 [1.34140953]]

Process finished with exit code 0

\end{python}