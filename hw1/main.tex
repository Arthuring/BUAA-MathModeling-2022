 % !TEX root = main.tex

 \documentclass[fontset=fandol]{ctexrep}

 \newcommand{\artdate}{2022年9月25日}
 
 \usepackage[toc,page]{appendix}
 \usepackage{graphicx}
 \usepackage{geometry}
 \usepackage{lscape}
 \usepackage{hyperref}
 \usepackage{hyperxmp}
 \usepackage{longtable}
 \usepackage{booktabs}
 \usepackage[bottom]{footmisc}
 \usepackage{subcaption}
 \usepackage{url}
 \usepackage{bytefield}
 \usepackage{xcolor}
 \usepackage{minted}
 \usepackage{threeparttable}
 \usepackage{multirow}
 \usepackage{float}
 \usepackage{amsmath}
 \usepackage{pythonhighlight}


 \newcommand{\artauthor}{}
 \newcommand{\arttitle}{数学建模-第1次作业-龚悦-20373091}

 \newcommand{\todo}{\textbf{\textcolor{red}{此部分尚需完善。}}}
 
 \hypersetup{pdfauthor=\artauthor,
             pdftitle=\arttitle,
             pdfdate=\artdate,
             pdfdisplaydoctitle=true,
             pdflang=zh-CN,
             pdfstartview=FitH,
             colorlinks=true,
             linkcolor=blue
             }
 
 \geometry{
     a4paper,
     left=3.18cm,
     right=3.18cm,
     top=2.54cm,
     bottom=2.54cm
 }
 
 \graphicspath{{figures/}}
 
 \ctexset{chapter={
         name={习题,},
         number=\chinese{chapter},
     },
     listfigurename=插图索引,
     listtablename=表格索引
 }
 
 \setminted{frame=lines, linenos=true, breaklines=true}
 
 \begin{document}
 
 \begin{titlepage}
 
     \newcommand{\HRule}{\rule{\linewidth}{0.5mm}}
   
     \vfill
     \center 
     
     \textit{\Large 北京航空航天大学}\\[0.5cm] 
     \texttt{\Large 计算机学院}
   
     \vspace{3 cm}
     \HRule \\[0.4cm]
     { \huge \bfseries 数学建模}\\[0.4cm]
     { \huge \bfseries 第一次作业}\\
     \HRule \\[1cm]
 
     \vspace{2.5 cm}
     龚悦\\
     学号:20373091\\
     班级:200615\\
 
     \vspace{1 cm}
     {\large \artdate}\\[3cm] 
   
   \vfill
   
 \end{titlepage}

 \tableofcontents
 \chapter{}

用多种(至少两个)方法求解差分方程:
\begin{equation*}
    \begin{split}
        &2x_{n+2} - x_{n+1} - 2x_n = 0\\
        &x_0 = -2\\
        &x_1 = 0
    \end{split}
\end{equation*}

\chapter{}
求下列微分方程的解析解和数值解,并画出曲线图,曲线图范围$t\in[0,1]$。
\begin{equation*}
    \left \{
        \begin{aligned}
            &\frac{\partial x}{\partial t} = x - 2y\\
            &\frac{\partial y}{\partial t} = x + 2y\\
            &x(0)=1, y(0) = 0
        \end{aligned}
    \right . 
\end{equation*}
 \appendix

  \chapter{代码附录}
\section{第一问}
\subsection{小数据}
\begin{python}
# -- coding: utf-8 --
import math

import numpy as np
import cvxpy as cp

K = 30
N = 42

# 读取文件中的点
points = np.genfromtxt("points_xyz_" + str(N) + ".txt", dtype=float)

# 计算点之间的覆盖半径
d = np.zeros((N, N))
for i in range(N):
    for j in range(N):
        cos = np.round(points[i].dot(points[j]) / (np.linalg.norm(points[i]) * np.linalg.norm(points[j])), 4)
        d[i, j] = np.arccos(cos)

# 定义变量 h可取0,1, 取1表示选择该点, 取0表示不选该点
h = cp.Variable(N, boolean=True)
theta = cp.Variable(1, pos=True)

# 优化目标
object = cp.Maximize(theta[0])

# h可取0,1
for i in range(N):
    conds.append(h[i] >= 0)
    conds.append(h[i] <= 1)

# theta >= d[i, j]
for i in range(N):
    for j in range(i + 1, N):
        conds.append(theta[0] - (1 - h[i]) * math.pi - (1 - h[j]) * math.pi <= d[i, j])

# 求解
prob = cp.Problem(object, conds)
prob.solve(solver='GUROBI')

# 输出
ans = np.nonzero(h.value)
theta_max = theta[0].value

print("选择点:", ans)
print("最优值:", theta_max)

\end{python}

\subsection{大数据}
\begin{python}
    # -- coding: utf-8 --
import math

import numpy as np
import cvxpy as cp
import gurobipy as gp

gp.setParam("TimeLimit", 600)
gp.setParam("MIPFocus", 1)


K = 30
N = 162

# 读取文件中的点
points = np.genfromtxt("points_xyz_" + str(N) + ".txt", dtype=float)

# 计算点之间的覆盖半径
d = np.zeros((N, N))
for i in range(N):
    for j in range(N):
        cos = np.round(points[i].dot(points[j]) / (np.linalg.norm(points[i]) * np.linalg.norm(points[j])), 4)
        d[i, j] = np.arccos(cos)

# 定义变量 h可取0,1, 取1表示选择该点, 取0表示不选该点
h = cp.Variable(N, boolean=True)
theta = cp.Variable(1, pos=True)

# 优化目标
object = cp.Maximize(theta[0])

# 限制条件 总点数为k,theta范围[0,pi]
conds = [sum(h) == K,
         theta[0] >= 0,
         theta[0] <= math.pi]
# h可取0,1
# for i in range(N):
#     conds.append(h[i] >= 0)
#     conds.append(h[i] <= 1)

# theta >= d[i, j]
for i in range(N):
    for j in range(i + 1, N):
        conds.append(theta[0] - (1 - h[i]) * math.pi - (1 - h[j]) * math.pi <= d[i, j])

# 求解
prob = cp.Problem(object, conds)
prob.solve(solver='GUROBI', verbose=True, MIPFocus=1)

# 输出
ans = np.nonzero(h.value)
theta_max = theta[0].value

print("选择点:", ans)
print("最优值:", theta_max)

\end{python}

\section{第二问}
\begin{python}
    # -- coding: utf-8 --
import numpy as np
import cvxpy as cp
import gurobipy as gp
gp.setParam("TimeLimit", 60)
gp.setParam("MIPFocus", 3)


S = 2
K = [10, 15]
w = 0.5

N = 162

# 读取文件中的点
points = np.genfromtxt("points_xyz_" + str(N) + ".txt", dtype=float)

# 计算点之间的覆盖半径
d = np.zeros((N, N))
for i in range(N):
    for j in range(N):
        cos = np.round(points[i].dot(points[j]) / (np.linalg.norm(points[i]) * np.linalg.norm(points[j])), 4)
        d[i, j] = np.arccos(cos)

# h[i][j] = 1 表示第j号点选入第i组   h[i][j] = 0表示第j号点不选入第i组
h = cp.Variable((S, N), boolean=True)
# theta[0, S-1]为各组的最小覆盖半径,theta[S]是所有点的最小覆盖半径
theta = cp.Variable(S + 1, pos=True)

object = cp.Maximize(w / S * sum(theta[:-1]) + (1 - w) * theta[S])
conds = []
# theta 范围
for i in range(S + 1):
    conds.append(theta[i] >= 0)
    conds.append(theta[i] <= np.pi)
# 每组选择的点的数量
for i in range(S):
    conds.append(cp.sum(h[i]) == K[i])
# 一个点只能在一个组
for i in range(N):
    conds.append(cp.sum(h[:, i]) <= 1)

# 每组的theta大于等于组中各点间距
for i in range(S):
    for j in range(N):
        for k in range(j+1,N):
            conds.append(theta[i] - (1 - h[i, j]) * np.pi - (1 - h[i, k]) * np.pi <= d[j, k])

# 总体的theta大于等于各点间距
for i in range(N):
    for j in range(i + 1, N):
        conds.append(theta[S] - (1 - cp.sum(h[:, i])) * np.pi - (1 - cp.sum(h[:,j])) * np.pi <= d[i, j])

prob = cp.Problem(object, conds)
prob.solve(solver='GUROBI')

h_value = h.value
theta_value = theta.value

for i in range(S):
    print("\n第", i+1, "组选择:")
    for j in range(N):
        if h_value[i][j] == 1:
            print(j+1, " ", end="")

print("总最优值:", prob.value)
\end{python}
 \end{document}
 