\chapter{代码附录}
\section{习题一}
\subsection{sympy代码}
\begin{python}
    # -- coding: utf-8 --
import sympy as sp
import numpy as np
# 定义变量n
n = sp.var('n', positive=True, integer=True)
# 定义转移矩阵L
L = sp.Matrix([[sp.Rational(1,2) , 1],
               [1, 0]])
# 初始条件
x_0 = sp.Matrix([[0], [-2]])
# 求出L的特征值
val = L.eigenvals()
print("特征值:{}".format(val))
# 求出L的特征向量
vec = L.eigenvects()
print("特征向量:{}".format(vec))
# 相似对角化L
P, D = L.diagonalize()
print("相似对角化后P:")
print(P)
print("相似对角化后D:")
print(D)
# 计算L的n次方
Lk = P @ (D ** n) @ (P.inv())
# 带入初始值,求出最终解
F = Lk @ sp.Matrix([[0], [-2]])
print("最终解为:\n{}".format(F))
\end{python}

\subsection{输出}
\begin{python}
    特征值:{1/4 - sqrt(17)/4: 1, 1/4 + sqrt(17)/4: 1}
特征向量:[(1/4 - sqrt(17)/4, 1, [Matrix([
[1/4 - sqrt(17)/4],
[               1]])]), (1/4 + sqrt(17)/4, 1, [Matrix([
[1/4 + sqrt(17)/4],
[               1]])])]
相似对角化后P:
Matrix([[1/4 - sqrt(17)/4, 1/4 + sqrt(17)/4], [1, 1]])
相似对角化后D:
Matrix([[1/4 - sqrt(17)/4, 0], [0, 1/4 + sqrt(17)/4]])
最终解为:
Matrix([[-2*(1/4 - sqrt(17)/4)*(1/4 - sqrt(17)/4)**n/(17/8 - sqrt(17)/8) - 2*(1/4 + sqrt(17)/4)*(1/4 + sqrt(17)/4)**n*(1/2 - sqrt(17)/34)], [-2*(1/4 - sqrt(17)/4)**n/(17/8 - sqrt(17)/8) - 2*(1/4 + sqrt(17)/4)**n*(1/2 - sqrt(17)/34)]])

\end{python}
\section{习题二}
\subsection{代码}
\begin{python}
# -- coding: utf-8 --
import sympy as sp
from scipy.integrate import odeint
import numpy as np
import pylab as plt

plt.rcParams['font.sans-serif'] = ['SimHei']
plt.rcParams['axes.unicode_minus'] = False  # 用来正常显示负号

# -----------求解析解-----------
# 定义自变量和函数
t = sp.var('t')
x = sp.Function('x')
y = sp.Function('y')
# 列出微分方程组
eq = (sp.Eq(sp.Derivative(x(t), t, 1), x(t) - 2 * y(t)), sp.Eq(sp.Derivative(y(t), t, 1), x(t) + 2 * y(t)))
# 列出初始条件
con = {y(0): 0, x(0): 1}
# 求解析解
result = sp.dsolve(eq, ics=con)
print(result)
sx = sp.lambdify(t, result[0].args[1], 'numpy')
sy = sp.lambdify(t, result[1].args[1], 'numpy')

# -----------求数值解-----------
dz = lambda z, t: [z[0] - 2 * z[1], z[0] + 2 * z[1]]
t = np.linspace(0, 1, 100)
s = odeint(dz, [1, 0], t)

# -----------画图-----------
# 画x-t图
plt.plot(t, sx(t), 'g-')
plt.plot(t, s[:, 0], '.')
plt.xlabel('$t$')
plt.ylabel('$x$', rotation=0)
plt.legend(['[0,1]上$x(t)$的解析解', '[0,1]上$x(t)$的数值解'])
plt.show()
# 画y-t图
plt.plot(t, sy(t), 'g-')
plt.plot(t, s[:, 1], '.')
plt.xlabel('$t$')
plt.ylabel('$y$', rotation=0)
plt.legend(['[0,1]上$y(t)$的解析解', '[0,1]上$y(t)$的数值解'])
plt.show()
# 画y-x图
plt.title('y(x)图')
plt.plot(s[:, 0], s[:, 1])
plt.xlabel('$x$')
plt.ylabel('$y$', rotation=0)
plt.show()

\end{python}

\subsection{输出}
\begin{python}
    [Eq(x(t), -sqrt(7)*exp(3*t/2)*sin(sqrt(7)*t/2)/7 + exp(3*t/2)*cos(sqrt(7)*t/2)), Eq(y(t), 2*sqrt(7)*exp(3*t/2)*sin(sqrt(7)*t/2)/7)]
\end{python}